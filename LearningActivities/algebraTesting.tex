\documentclass{ximera}

\input{../xmpreamble.tex}

\title{Radicals and Rational Exponents}
\author{College Algebra Practice}

\begin{document}
\begin{abstract}
Practice problems involving simplifying radicals, operations with radicals, and rational exponents.
\end{abstract}
\maketitle

\section{Simplifying Radicals}

\begin{problem}
Simplify $\sqrt{72}$.

$\sqrt{72} = \answer{6\sqrt{2}}$

\begin{hint}
Factor 72 into perfect square factors: $72 = 36 \cdot 2 = 6^2 \cdot 2$
\end{hint}
\end{problem}

\begin{problem}
Simplify $\sqrt[3]{54}$.

$\sqrt[3]{54} = \answer{3\sqrt[3]{2}}$

\begin{hint}
Factor 54: $54 = 27 \cdot 2 = 3^3 \cdot 2$
\end{hint}
\end{problem}

\begin{problem}
Simplify $\sqrt{x^8y^5}$ assuming all variables are positive.

$\sqrt{x^8y^5} = \answer{x^4y^2\sqrt{y}}$

\begin{hint}
Use the property $\sqrt{a^n} = a^{n/2}$ when the exponent is even.
\end{hint}
\end{problem}

\section{Operations with Radicals}

\begin{problem}
Simplify $3\sqrt{12} + 5\sqrt{27}$.

$3\sqrt{12} + 5\sqrt{27} = \answer{21\sqrt{3}}$

\begin{hint}
First simplify each radical: $\sqrt{12} = 2\sqrt{3}$ and $\sqrt{27} = 3\sqrt{3}$
\end{hint}
\end{problem}

\begin{problem}
Multiply and simplify $\sqrt{6} \cdot \sqrt{15}$.

$\sqrt{6} \cdot \sqrt{15} = \answer{3\sqrt{10}}$

\begin{hint}
Use the property $\sqrt{a} \cdot \sqrt{b} = \sqrt{ab}$, then factor and simplify.
\end{hint}
\end{problem}

\begin{problem}
Rationalize the denominator: $\frac{5}{\sqrt{3}}$.

$\frac{5}{\sqrt{3}} = \answer{\frac{5\sqrt{3}}{3}}$

\begin{hint}
Multiply both numerator and denominator by $\sqrt{3}$.
\end{hint}
\end{problem}


%another problem but changing a radical involving 9 and another prime

\begin{problem}
Rationalize the denominator: $\frac{2}{3 + \sqrt{9}}$.

$\frac{2}{3 + \sqrt{9}} = \answer{\frac{3 - \sqrt{9}}{2}}$
\end{problem}

\begin{problem}
Rationalize the denominator: $\frac{2}{3 + \sqrt{5}}$.

$\frac{2}{3 + \sqrt{5}} = \answer{\frac{3 - \sqrt{5}}{2}}$


\begin{hint}
Multiply by the conjugate $\frac{3 - \sqrt{5}}{3 - \sqrt{5}}$ and use the difference of squares formula.
\end{hint}
\end{problem}

\section{Rational Exponents}

\begin{problem}
Write $\sqrt[4]{x^3}$ using rational exponents.

$\sqrt[4]{x^3} = \answer{x^{3/4}}$

\begin{hint}
The $n$th root of $x^m$ is $x^{m/n}$.
\end{hint}
\end{problem}

\begin{problem}
Simplify $8^{2/3}$.

$8^{2/3} = \answer{4}$

\begin{hint}
$8^{2/3} = (8^{1/3})^2 = 2^2 = 4$
\end{hint}
\end{problem}

\begin{problem}
Simplify $27^{-2/3}$.

$27^{-2/3} = \answer{\frac{1}{9}}$

\begin{hint}
$27^{-2/3} = \frac{1}{27^{2/3}} = \frac{1}{(27^{1/3})^2} = \frac{1}{3^2}$
\end{hint}
\end{problem}

\begin{problem}
Simplify $(x^{1/4})^8$ assuming $x > 0$.

$(x^{1/4})^8 = \answer{x^2}$

\begin{hint}
Use the power rule: $(x^m)^n = x^{mn}$
\end{hint}
\end{problem}

\section{Solving Radical Equations}

\begin{problem}
Solve $\sqrt{x + 3} = 5$.

$x = \answer{22}$

\begin{hint}
Square both sides: $x + 3 = 25$, then solve for $x$.
\end{hint}
\end{problem}

\begin{problem}
Solve $\sqrt{2x - 1} = x - 2$.

The solution is $x = \answer{5}$.

\begin{hint}
Square both sides, solve the resulting quadratic, and check for extraneous solutions.
\end{hint}
\end{problem}

\begin{problem}
Solve $\sqrt{x + 1} + \sqrt{x - 3} = 4$.

$x = \answer{8}$

\begin{hint}
Isolate one radical, square both sides, then repeat the process. Check your answer!
\end{hint}
\end{problem}

\section{Challenge Problems}

\begin{problem}
If $\sqrt{x} + \sqrt{y} = 7$ and $x + y = 25$, find $xy$.

$xy = \answer{12}$

\begin{hint}
Square the first equation and use the second equation to eliminate terms.
\end{hint}
\end{problem}

\begin{problem}
Simplify $\frac{\sqrt{a^3b^5}}{\sqrt{ab}}$ where $a, b > 0$.

$\frac{\sqrt{a^3b^5}}{\sqrt{ab}} = \answer{ab^2}$

\begin{hint}
Use the property $\frac{\sqrt{x}}{\sqrt{y}} = \sqrt{\frac{x}{y}}$ and simplify the fraction under the radical.
\end{hint}
\end{problem}

\end{document}